% Apostila sobre programação Web
% Utilizamos o template Latex disponível aqui:
% http://www.ime.usp.br/~tassio/arquivo/latex/apostila.pdf

    \documentclass[%draft,
    a4paper,11pt,twoside]{article}
    \input{head}
    \title{\LaTeX ação\\{\footnotesize Programador Web}}
    \author{Vinícius Alves Hax} % por enquanto
    \hypersetup{baseurl={http://www.ime.usp.br/~tassio/apostila.pdf},
    pdftitle={Programador Web},
    pdfauthor={Vinícius Alves Hax},
    pdfkeywords={PHP, HTML, CSS, Programação, Web},
    pdflang={pt-BR (Portuguese)},
    unicode=true}
    \begin{document}
    \maketitle
    \thispagestyle{empty}
    \clearpage
%    \input{00-sobre}
    \clearpage
%    \input{thanks}\clearpage
    %\begin{footnotesize}
    \tableofcontents
    %\end{footnotesize}
    \clearpage
% Informática Básica
%    \input{01-informatica}
% Implementação de Interface Web
%    \input{02-fluxo-trabalho}
% Lógica de Programação
%    \input{03-primeiro-texto}
% Implementação de sites e aplicativos
%    \input{04-estruturando-o-texto}
    \newpage
    \appendix
%    \input{fdl-1.3}
    \newpage
    \printindex%
    \addcontentsline{toc}{section}{Índice Remissivo}
    \newpage
    \bibliographystyle{babalpha}
    \cleardoublepage \phantomsection
    \addcontentsline{toc}{section}{Referências}
    \bibliography{thebib}
    \end{document}

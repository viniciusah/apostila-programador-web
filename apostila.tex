% Apostila sobre programação Web
% Utilizamos o template Latex disponível aqui:
% http://www.ime.usp.br/~tassio/arquivo/latex/apostila.pdf

    \documentclass[%draft,
    a4paper,11pt,twoside]{article}
%    \input{head}
    \title{\LaTeX ação\\{\footnotesize Programador Web}}
    \author{Vinícius Alves Hax} % por enquanto
    \hypersetup{baseurl={http://www.ime.usp.br/~tassio/apostila.pdf},
    pdftitle={Programador Web},
    pdfauthor={Vinícius Alves Hax},
    pdfkeywords={PHP, HTML, CSS, Programação, Web},
    pdflang={pt-BR (Portuguese)},
    unicode=true}
    \begin{document}
    \maketitle
    \thispagestyle{empty}
    \clearpage
%    \input{00-sobre}
    \clearpage
%    \input{thanks}\clearpage
    %\begin{footnotesize}
    \tableofcontents
    %\end{footnotesize}
    \clearpage
% Informática Básica
%    \input{01-informatica}
% Implementação de Interface Web
    \section{Implementação de interfaces web}

\subsection{Arquitetura da Internet}

\subsection{Introdução a HTML}

\subsection{Imagens}

\subsection{Listas}

\subsection{Caminhos absolutos}

\subsection{Comentários}

\subsection{Link}

\subsection{Tags h1 h6}

\subsection{Tabelas}

\subsection{CSS}

\subsection{Tableless}

\subsection{CSS Box Model}

\subsection{Tags INPUT}

\subsection{Posicionamento em HTML com CSS}

\subsection{Exercício 1}
Exercício sobre desenvolvimento de páginas

\subsection{Exercício 2}
Exercício sobre desenvolvimento de páginas: reproduzir http://www.csszengarden.com/210/

\subsection{Construindo um site a partir de uma imagem}
Exercício construção de site a partir de imagem

\subsection{Ferramentas úteis na construção de sites}
W3C Validator, W3C Schools

\subsection{CSS3}

% Lógica de Programação
    \section{Lógica de Programação}

\subsection{Introdução a programação de computadores}

\subsection{Pseudo-linguagem}

\subsection{Diagrama de blocos}

\subsection{Variáveis}

\subsection{Estruturas de seleção}

\subsection{Operadores lógicos}

\subsection{Operadores condicionais}

\subsection{Estruturas de repetição enquanto e faça-enquanto}

\subsection{Estrutura de repetição do tipo faça (for)}

\subsection{Listas}

\subsection{Introdução a programação com PHP}

\subsection{Estruturas de seleção com PHP}

\subsection{Estruturas de repetição com PHP}

\subsection{Listas em PHP}

% Implementação de sites e aplicativos
    \section{Implementação de Sites e Aplicativos}

\subsection{Introdução a banco de dados}
\subsection{Passagem de parâmetros por GET e POST}
\subsection{Implementação CRUD: Read e Create}
\subsection{Implementação CRUD: Update e Delete}
\subsection{Implementação CRUD: Update e Delete}
\subsection{Sessões em PHP}
\subsection{Implementação de sistema de login com senha}
\subsection{Diretiva include, require e organização de código}
\subsection{Implementação de sistema de autorização: verificação de objetos públicos}
\subsection{Implementação de sistema de autorização: verificação de donos dos objetos}
\subsection{Integrando PHP com HTML e CSS}
\subsection{Implementação de upload de imagem}
\subsection{Integrando o upload de imagens com nosso sistema}

    \newpage
    \appendix
%    \input{fdl-1.3}
    \newpage
    \printindex%
    \addcontentsline{toc}{section}{Índice Remissivo}
    \newpage
    \bibliographystyle{babalpha}
    \cleardoublepage \phantomsection
    \addcontentsline{toc}{section}{Referências}
    \bibliography{thebib}
    \end{document}

\section{Implementação de interfaces web}

\subsection{Arquitetura da Internet}

\subsection{Introdução a HTML}
Começamos nosso estudo pelo elemento básico responsável pela Internet: o HTML. O HTML é uma linguagem tal qual o português, porém mais simples e possível de ser interpretada por computadores. Como outras linguagens ela possue as suas próprias regras e palavras. Aprenderemos mais sobre as regras do HTML ao longo do curso e aprenderemos as palavras mais comuns do HTML. As palavras no HTML são chamadas de \textit{tags}.

Quando escrevemos uma sequência de palavras devemos tomar cuidado para que essa sequência faça sentido, da mesma maneira que no português. Porém além de fazer sentido para nós, seres humanos, essa sequência de \textit{tags} precisa fazer sentido para o computador também. Na verdade o computador, através de um \textit{software} específico é que irá ler muitas e muitas vezes os nossos arquivos HTML.

Cada \textit{tag} em HTML começa com o símbolo < e termina com o símbolo >. As \textit{tags} em HTML geralmente são usadas aos pares. Em função disso existem três tipos de \textit{tags} em HTML:
As \textit{tags} de abertura que tem o seguinte formato <NOME>.
As \textit{tags} de fechamento que tem o seguinte formato </NOME> e por fim as \textit{tags} que são ao mesmo tempo de abertura e fechamento. Essas últimas não são usadas aos pares e de maneira geral possuem o seguinte formato <NOME />. Logo a seguir veremos exemplos dos três tipos.

As tags também podem possuir um ou mais atributos e cada atributo pode ter um valor. As \textit{tags} com atributos tem o seguinte formato:

<NOME atributo1="valor1" atributo2="valor2 />

Podemos pensar nos atributos das \textit{tags} como propriedades delas que podem ter seus valores alterados. Supondo que existisse uma \textit{tag} pessoa um possível atributo dessa \textit{tag} seria cor dos olhos, e poderíamos citar como valores possíveis azul ou castanho. Entre um atributo e o seu valor utilizamos o sinal de = para simbolizar que um determinado atributo passou a ter um valor igual ao valor informado. Observem que quando for informado um valor o mesmo deverá estar entre aspas. Quando formos conhecendo as \textit{tags} iremos conhecer também os atributos e valores mais comuns para as \textit{tags}.

\subsection{Link}

\subsection{Caminhos absolutos}

\subsection{Imagens}

\subsection{Listas}

\subsection{Comentários}

\subsection{Tags h1 h6}

\subsection{Tabelas}

\subsection{Tags INPUT}

\subsection{CSS}

\subsection{Tableless}

\subsection{CSS Box Model}

\subsection{Posicionamento em HTML com CSS}

\subsection{Exercício 1}
Exercício sobre desenvolvimento de páginas

\subsection{Exercício 2}
Exercício sobre desenvolvimento de páginas: reproduzir http://www.csszengarden.com/210/

\subsection{Construindo um site a partir de uma imagem}
Exercício construção de site a partir de imagem

\subsection{Ferramentas úteis na construção de sites}
W3C Validator, W3C Schools

\subsection{CSS3}
